\chapter{Java}

Samotný vykonávaci program je napisaný v jazyku Java. Na spustenie je potrebné mať nainštalovanú minimáalne verziu Java 1.7.
Jeho úlohou je na základe dodaných údajov v dohodnutom interface pripravit a vygenerovať obrázok v zmysle štandaru Slovenskej bankovej asociácie.

QR kód pay by square môže obsahovať údaje pre nasledovné platobné príkazy:
\begin{itemize}
    \item platobný príkaz
    \item hromadný platobný príkaz
    \item trvalý platobný príkaz
    \item zriadenie inkasa
\end{itemize}

Program má v súčasnosti implementovanú len prvú možnosť, teda platobný príkaz pre 1 účet.

\section{Interface}

V tabuľke \ref{tab:java_interface} je zobrazená štruktúra tabuľky. 

\begin{table}[!ht]
\caption{Interface Java programu pay by square}
\label{tab:java_interface}
\begin{center}
    \begin{tabular}{|c|c|c|l|}
    \hline
    \textbf{Názov} & \textbf{Datový typ} & \textbf{Povinné pole} & \textbf{Popis} \\ \hline
    HCIIV           & BigDecimal & Áno & číslo faktúry, pre internú identifikáciu \\ \hline
    HCIAM           & BigDecimal & Áno & suma \\ \hline
    HDUT            & String     & Nie & dátum platby \\ \hline
    VARSYM          & String     & Nie & variabilný symbol, ak je prázdny použije sa pole HCCIV\\ \hline
    HSSYM           & String     & Nie & špecifický symbol \\ \hline
    HPAYN           & String     & Áno & správa pre príjmateľa \\ \hline
    HIBAN           & String     & Áno & IBAN číslo účtu príjmateľa \\ \hline
    HBIC            & String     & Áno & SWIFT kód banky príjmateľa \\ \hline
    CurrCode        & String     & Áno & konštanta 'EUR' \\ \hline
    RefInf          & String     & Áno & V tvare /VS[VARSYM]/SS[HSSYM]/KS[308] \\ \hline
    BenName         & String     & Áno & konštanta 'Ayvens' \\ \hline
    BenAdr          & String     & Áno & konštanta 'Ševčenkova 34,851 01 Bratislava'\\ \hline
    QRFilePath      & String     & Áno & konštanta '' \\ \hline
    QRFileName      & String     & Áno & názov súboru v tvare 'HCIIV.png' \\ \hline
    \end{tabular} 
\end{center}    
\end{table}

\newpage
\section{Popis programu}

Program je impelemtnovaný v hlavnej triede PayBySquareGen, ktorá implentuje interface PayBySquareGen_I. Trieda poskytuje pulic metódu  generate, ktora je override z interface PayBySquareGen_I. Ide o metódu so vstupnými parametrami v zmysle dohodnotého interface medzi AS400 a týmto Java programom. Java program umožnuje vygenerovanie QR kódu aj na základe parametrov, ktoré sa odovzdávaju ako commandline parametre.

Program na generovanie QR kódu pay by square prebieha v nasledujúcich krokoch:

\begin{enumerate}
    \item vygenerovanie textoveho reťazca na základe zadaných údajov z AS400
    \item výpočet CRC32 kontrolnej sumy
    \item LZMA kompresia
    \item Base32 kódovanie
    \item vygenerovanie QR kódu a uloženie do požadovaného súboru
\end{enumerate}


\section{Logovanie}
Logy sú generované pomocou knižnice log4j a sa zaznamenaju buď fyzicky do súboru alebo môžu byť nasmerované aj do konzoly. Súbor sa ukladá do adresára '/logs/'. Obsahujú niekoľko typov tzv. levelov logovania Info, Debug, Error. 
Konfigúracia kde sa toto všetko nastavuje sa nachádza v súbore log4j.properties, kde sa na začiatku definguje tzv. root logger options. Ďalej nasledujú tzv. appenders, cez ktoré si vieme zmeniť napr aj adresár a názov súboru, max. veľkosť súboru prípadne max. počet súborov... Ďalej sa da ovplývniť aj formát vygenerovaného logu.
